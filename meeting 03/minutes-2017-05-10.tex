\documentclass[a4paper,twoside,11pt]{article}
\usepackage{a4wide,graphicx,fancyhdr,amsmath,amssymb,float,wrapfig}
\usepackage[bottom]{footmisc} % Make footnotes stick to the bottom of a page
\usepackage{hyperref}
\usepackage{lineno}


%----------------------- Macros and Definitions --------------------------

%\setlength\headheight{20pt}
\addtolength\topmargin{0pt}
%\addtolength\footskip{20pt}

\fancypagestyle{plain}
\fancyhf{}
\renewcommand{\headrulewidth}{0pt}
\renewcommand{\footrulewidth}{0pt}

\pagestyle{fancy}
\fancyhf{}
\renewcommand{\headrulewidth}{0pt}
\renewcommand{\footrulewidth}{0pt}

%
\newcommand{\ts}{\textsuperscript}




%-------------------------------- Title ----------------------------------

\title{}
\date{}

%--------------------------------- Text ----------------------------------

\begin{document}

%\maketitle



\begin{center}
\Huge{Minutes 3\ts{rd} Meeting Group 14}
\\\vspace*{2mm}
\Large{May 10 2017}
\\\vspace*{2mm}
\large{MF 5.146}
\\
		\end{center}

		\textbf{Present:}  \textit{Bogdan Floris} (chairman) , \textit{Jelle Wemmenhove} (secretary), \textit{Mike de Brouwer},  
\\\indent\qquad\,\,\,\,\,\qquad\quad \textit{Sergiu Marin}, \textit{Job Savelsberg}, \textit{Martijn Struijs} (tutor)


	    \textbf{Absent:} -\\
	    
	    \hline

\linenumbers
\modulolinenumbers[5]

	\section{Opening}
	
	Bogdan opens the meeting at 16:59 hour.
	
	\subsection{General remarks}
	
	- 
	
	\section{The state of the implementation}
	
	Sergiu and Bogdan are separately working on two greedy implementations.
	
	Sergiu has started implementing a greedy algorithm that should closely approximate an optimal solution. The main idea is that it places a rectangle and then restricts itself to the largest rectangle of remaining empty space, wasting a small part of free space. He plans to have it implemented by Friday (May 12\ts{th}). The tutor Martijn mentions that another greedy approach can be used in hopes of finding higher quality solutions: the greedy algorithm can split up the problem into smaller subproblems for which a optimal solution can be solved. 
	
	Bogdan's greedy algorithm is aimed at speed. It places the rectangles in order of largest to smallest area and places them as close to the left-bottom corner as possible. He plans to finish the implementation by this weekend (May 15\ts{th}).
	
	
	\section{The state of the GUI}
	
	Job has made a simple program that visualizes the output generated by the packing algorithm. It show the placement and rotation of the rectangles which are all shown by a different colour. The group discusses what can still be added. It is decided that the GUI should be able to calculate some performance measures such as the percentage of wasted space. Furthermore we should be able to use the GUI to generate test cases by hand (for example by dragging rectangles over the screen). 
	
	The subject of the discussion changes to test cases and their generation. Jelle mentions that the paper \textit{Optimal Rectangle Packing: An Absolute Placement Approach} by Huang and Korf extensively discusses several benchmarks. These offer a systematic approach to test the algorithms by having the shape and number of the rectangles depend on a single parameter $N$. Jelle advices Job to have a look at these.
	
	It is remarked that using general test cases is good, the algorithm should also be specialized to handle the test cases used to determine the final grade as well as possible. Canvas can be used to test this. Martijn warns the group not to rely too heavily on the testing system used in Canvas and take add a margin of error to the run time of our implementation. He also warns that an internal timer can be used, but that it is tricky to implement this correctly. 
	
	
	\section{Algorithm discussion}
	
	Sergiu and Jelle have looked into some algorithms. The paper mentioned above seems to be the most state-of-the-art. Sergiu advices Jelle to also look into the author's previous paper from 2010 as it specifies a lot of pseudo-code used, the new 2013 paper mainly mentions improvements made to this algorithm.
	
	
	\section{Project planning}
	
	It is decided that a meeting on Friday (May 12\ts{th}) will not be necessary.
	
	\subsection{Work division}
	
	\begin{itemize}
	    \item Sergiu and Bogdan will have implemented their algorithms before the next meeting.
	    \item Job and Mike will work on the GUI and will have made some test cases before the next meeting.
	    \item Jelle will start writing the report and continue reading papers.
	    \item Jelle will send the tutor a link to the Drive folder.
	\end{itemize}
	
	
	\section{Final inquiry}
	
		
    \section{Closing}
    
    Bogdan closes the meeting at 17:25 hour.
    
	
\end{document}