\documentclass[11pt]{article}

\usepackage{amsmath,amssymb}
\usepackage{a4wide}
\usepackage{graphicx}
\usepackage{clrscode3e}

\newcommand{\maxsize}[1]{\begin{quotation} {\sl \noindent Maximum size: #1.} \end{quotation}}

\newcommand{\ml}[1]{\begin{quotation} {\sl \noindent For the map-labeling problem: #1} \end{quotation}}

\newcommand{\example}[1]{\begin{quotation} {\sl \noindent Example: #1} \end{quotation}}

\newtheorem{defin}{Definition}
\newenvironment{mydefinition}{\begin{defin} \sl}{\end{defin}}
\newtheorem{theo}[defin]{Theorem}
\newenvironment{mytheorem}{\begin{theo} \sl}{\end{theo}}
\newtheorem{lem}[defin]{Lemma}
\newenvironment{lemma}{\begin{lem} \sl}{\end{lem}}
\newtheorem{coro}[defin]{Corollary}
\newenvironment{corollary}{\begin{coro} \sl}{\end{coro}}
\newtheorem{obse}[defin]{Observation}
\newenvironment{observation}{\begin{obse} \sl}{\end{obse}}

\newenvironment{proof}{\emph{Proof.}}{\hfill $\Box$ \medskip\\}


\title{A Solution to the Rectangle Packing Problem}
\author{M.A.J. de Brouwer \and B. Floris \and S.C.I. Marin \and J.W.L. Savelsberg \and A.J. Wemmenhove}
\date{\today}

\begin{document}

\maketitle

\begin{abstract}
In the abstract you give an overview---typically one short paragraph---of the contents of
your paper: you describe the problem you have studied, and what the main results are.
\end{abstract}

\section{Introduction}
\label{se:introduction}
\maxsize{2 pages}
The introduction usually starts with a description of and
a motivation for the general problem area.

\ml{Start by describing the map-labeling in general terms, discuss applications and
	different variants of the problem, and so on.}

	After introducing the general problem area, you zoom in to the specific problem
	studied in the paper. I like to already discuss previous work here. This way
	you can explain where the specific problem fits into the state-of-the-art
	and why it is interesting. Ideally, the discussion of the previous work
	culminates in a clear statement about what is still missing in the current
	state-of-the-art: namely an answer to the specific problem you study.
	%
	\ml{Here you state the variants of the problem that you study, and discuss the results
		(time bounds, theoretical guarantees on the quality of the results, experimental results,
		 if applicable) from the literature on these variants. You may also want to describe
			in a few lines the approaches that are used in some of the papers.}
			%
			Then you give an overview of your results.
			%
			\ml{Try to give a high-level description of the
				approach(es) you have used and relate them to approaches found in the literature.
					State the theoretical guarantees (on running time,
							for instance, or on other aspects) that you may have proved for your algorithms
					and mention the main conclusions from the experiments.}
					%
					Finally, you can give an overview of the structure of the rest of your paper.
					Personally, however, I do not find these overviews very useful: I prefer to
					integrate this with the previous part of the introduction, where the overview
					of the results is given.
					\medskip

					\emph{Note:} I expect your introduction to
					contain five to ten references to papers describing related work.

					\section{The algorithms}
					\label{se:algorithms}
					\maxsize{8 pages. Use subsections where appropriate. Often it helps to use figures
						to illustrate definitions and concepts used in the algorithm}
						%
						The description of the algorithms should be such that a programmer can
						implement them without much difficulty. It is good practice to first explain the
						main ideas behind the algorithm at a more intuitive level, and then give a
						detailed description (for example using pseudo-code).
						Don't forget to describe which supporting data structures you use:
						linked lists, arrays, search trees, and so on. For standard data structures
						from the literature you do not need to explain how they work; a reference
						to the literature suffices.
						%
						\example{We store the set $P$ of points in a red-black tree~\cite{clrs-ia-01},
							using the $x$-coordinate of each point as its key.}
							%
							Try to theoretically analyze the worst-case running time of your algorithms
							and the amount of storage they use, and argue that it satisfies the requirements.
							Also try to say something about the quality of your algorithm: you might be able to prove that the algorithm is guaranteed to find an optimal solution, or
							you might be able to prove that the result of your algorithm is within a certain factor
							from optimum. (\emph{Note:} If you are using an algorithm from the literature, you should not
									copy the proofs from the paper describing the algorithm. Instead, give a short sketch that
									summarizes the main ideas.)
							It is also useful to show example for which the algorithm does not
							perform well.


							\section{Experimental evaluation}
							\label{se:evaluation}
							\maxsize{8 pages. Use subsections where appropriate}
							Describe the experiments, give the results of the experiments (in the form of tables or graphs),
							and discuss the results. Here you can also include pictures of your output for a few tests.
							Relate the outcome of the experiments to your theoretical analysis.

							When you investigate running times, you should say on which machine you ran the experiments.
							You may also want to investigate the running time in a  machine-independent manner,
							for example, by counting the number of calls to some basic procedure.
							It is important that you describe carefully how you generated the data sets.
							\ml{Possible topics to investigate include: How does the running time of your algorithm
								scale when the number of points increases? Compare the height of the labels achieved by
									your algorithms for the different placement models. Are the results influenced by the
									distribution of the input points? How far from optimal are your results? (For the latter you
											would need to know an optimal solution. For this, you may be able to generate your test instances in
											such a way that the maximum label height is known. If your algorithm contains certain
											parameters: investigate how the values for the parameters influence the results.}

											Note that the experimental data in itself are not the main result: the discussion
											and the conclusions you can draw are what makes the data interesting.





											\section{Concluding remarks}
											\label{se:conclusions}
											\maxsize{1 page}
											Give a short overview of the main results, discussing both the strong
											points of your algorithms as well as their weak points, and ideas for improvements
											(``future work'').

											\bibliographystyle{plain}

									\begin{thebibliography}{}

									\bibitem{a-raoa-02}
									S. Albers.
									On randomized online scheduling.
									In \emph{Proc. 34th ACM Symp. Theory Comput.}, pages 134--143, 2002.

									\bibitem{clrs-ia-01}
									T.H. Cormen, C.E. Leiserson, R.L. Rivest and C. Stein.
									\emph{Introduction to Algorithms} (2nd edition).
									MIT Press, 2001.

									\bibitem{m-apca-83}
									N. Megiddo.
									Applying parallel computation algorithms in the design of serial algorithms.
									\emph{J. ACM} 30: 852--865 (1983).

									\end{thebibliography}

									Make sure your references are well-polished and complete.
									References are typically to books and to papers in scientific
									journals and conference proceedings. References to web pages are hardly
									ever appropriate.
									The list of references is usually ordered alphabetically by first author (although some journals
											list them in the order they are cited for the first time in the paper).
									Only put in references that are actually cited in your paper.
									Formatting can be done as in the example above---note that these references have
									nothing to do with the topic of the DBL-project---, where the following
									formatting rules have been used:
									\begin{itemize}
									\item for journals:
									Authors. Title of paper. \emph{Journal Name (italic)} volume: page numbers (year).
									See reference~\cite{m-apca-83}.
									\item for conference proceedings:
									Authors. Title of paper. In \emph{Proc. Conference Name and number (italic)}, pages xxx--yyy, year.
									See reference~\cite{a-raoa-02}
									\item for books: Authors. \emph{Book title (italic)}. Publisher, year.
									See reference~\cite{clrs-ia-01}
									\end{itemize}
									Note that names of journals and conferences are usually abbreviated. There is a more
									or less standard way of doing this (for example, \emph{J. ACM} stands for \emph{Journal of the ACM},
											but how you do it exactly is less important than that
											you are consistent and list all the necessary information. Other things to consider:
											\begin{itemize}
											\item Make sure you are consistent in either writing first names in full (Susanne Albers)
											or only using initials (S.~Albers). The latter is preferred.
											Some people argue that one should write names in the same way as in
											the original publication. For example, when the title page of reference~\cite{a-raoa-02}
											would say ``Susanne Albers'', then you should also write the first name in full in
											your list of references. I find consistency more important, especially since
											the same author may appear in different ways on different papers. Moreover,
											writing first names in full can lead to long references, which is why
											initials are preferred.
											\item Sometimes the location where a conference took place, or where a publisher is based,
											is also given in the references. I find this useless.
											\item In theoretical computer science it is common to use numbers as labels when citing a paper.
											(In \LaTeX, use \verb#\bibliographystyle{plain}#.)
											\example{We store the set $P$ of points in a red-black tree~\cite{clrs-ia-01},
											using the $x$-coordinate of each point as its key.}
											In other areas it may be more
											common to use (author names + year of publication) as a label, as in
									\example{We store the set $P$ of points in a red-black tree~(Cormen et al. 2001),
										using the $x$-coordinate of each point as its key.}

										\end{itemize}

										\section*{Appendix A}
										Appendices are not allowed in the report for this DBL project.

										\end{document}
